%%%%%%齋藤研究室レジュメテンプレート%%%%%%% 2024

\documentclass[a4j]{jsarticle}
\usepackage[utf8]{inputenc}
\usepackage{amsmath,amssymb,amscd,amsthm}
\usepackage{ascmac,fancybox} 
\usepackage{bm}
\usepackage[dvipdfmx]{graphicx}
\usepackage{algorithm}%擬似コードを書く場合
\usepackage{algorithmic}%擬似コードを書く場合
%%%% ↓algotithmic の \REQUIRE と \ENSURE の表記を変更する
\renewcommand{\algorithmicrequire}{\textbf{Input:}}
\renewcommand{\algorithmicensure}{\textbf{Output:}}
%%%% ↑algotithmic の \REQUIRE と \ENSURE の表記を変更する
\def \QED{\hfill $\Box$}%証明終了の記号


%%%%%%%本文%%%%%%% 
\title{第2回\\学部3年前期ゼミナール発表資料}
\author{青山和樹}
\date{2024年4月15日}

\begin{document}

\maketitle

%%目的%%
\section*{発表の目的}
テキスト[1]の
\begin{itemize}
	\item 6~11ページの
	\begin{itemize}
		\item[C)] 集合の相等
		\item[D)] 部分集合
	\end{itemize}
	\item 11ページの問2
\end{itemize}
について発表する.

%%目次%%
\tableofcontents

\clearpage

%%本文%%
\section{集合の相等}
集合が,いくつかのものをひもまとめにして考えた「ものの集まり」であることから分かるように,集合はその元の全体によって完全に決定される.

\medskip

\begin{itembox}[l]{\textgt{定義 1.1} (集合の相等)}
集合$A, B$は,全く同じ元から成るとき,すなわち$A$の任意の元は同時にまた$B$の元でもあり,$B$の任意の元は同時にまた$A$の元でもあるとき,等しいという.そのとき$A=B$と書く.
\end{itembox}

\textgt{例 1.2}
\begin{align}
	\{x \mid xは2<x<10である素数\} = \{x \mid xは1<x<8である奇数\}\\
	\{x \mid x \in \mathbb{R},\ x^2+x-2<0 \} = \{x \mid x \in \mathbb{R},\ -2<x<1\}
\end{align}

\medskip

\textgt{解説 1.3} 外延的記法によれば,元$a, b, c, ...$から成りえながら,異なる集合$A, B$があるとき,$\{a, b, c, ...\}$と表してしまうと,$A, B$のどちらを示しているのか不明であるため注意する.また,集合は「元の全体」で決定されるため,書き示す際,その元の順序は任意に変えてもさしつけないし,重複した元があったとしてもその効果がただ一つだけ書いたのと同じことにも注意する.

\medskip

\textgt{例 1.4}
\begin{align}
	\{1,2,3,4\} = \{2,3,4,1\} = \{3,4,1,2\}\\
	\{1,2,3\} = \{1,2,2,3,3,3\}
\end{align}

\medskip

\textgt{解説 1.5} 前項の内包的表記において,ある変数$x$について条件$C(x)$があるときの集合\[ \{ x \mid C(x)\} \]と書くとき,変数記号$x$は$C(x)$の文章中に含まれないほかの任意の文字に置き換えることができる.

\medskip

\textgt{例 1.6} $y$を$C(x)$に含まれていない新しい変数記号とするなら\[ \{ x \mid C(x)\} = \{ y \mid C(y)\} \]と表せる.

\subsection{論理記号}

\begin{itembox}[l]{\textgt{定義 1.7} (含意、同値)}
論理記号$\Rightarrow$,$\Leftrightarrow$について,論理記号$\Rightarrow$は一般に二つの文章$p,q$が与えられたとき,$p \Rightarrow q$は「$p$が正しいときには$q$もまた正しい,ならば真」とされる.また$p \Rightarrow q$かつ$q \Rightarrow p$のとき$p \Leftrightarrow q$で表す.
\end{itembox}

\textgt{解説 1.8} 数学の証明においては,「pならばq」は「pでない,または,qである」と全く同じ意味で使われる.例えば,pが成立しないときは,qの成立/不成立に関わらず「pならばq」という論理式は成り立つのである.

\textgt{解説 1.9} 定義 1.1 集合の相等について,定義 1.7 の論理記号を使うことによって,\\
集合$A, B$があるとき,$A=B$とは,任意の対象$x$について$x \in A \Leftrightarrow x \in B$が成り立つときである\\
とも表せる.

\section{部分集合}

\begin{itembox}[l]{\textgt{定義 2.1} (部分集合)}
	集合$A, B$において$A$の元がすべてまた$B$の元でもあるならば,すなわち,任意の$x$について,$x \in A \Rightarrow x \in B $が成り立つならば,$A$は$B$の部分集合であるといい,
	\begin{equation}
		A \subset B または B \supset A
	\end{equation}
	とかく.その否定は
	\begin{equation}
		A \not\subset B または B \not\supset A
	\end{equation}
	と表す.
\end{itembox}

\textgt{例 2.2}
\begin{align}
	\{a, b\} \subset \{a, b, c\}\\
	\{1, 2 \} \not\subset \{2, 3\}
\end{align}

\medskip

\begin{itembox}[l]{\textgt{定義 2.3} (真部分集合)}
集合$A, B$において$ A \subset B でかつ A \neq B$のとき,とくに真部分集合であるという.
\end{itembox}

\textgt{解説 2.4} $1 \subset \{1\}$であるか否か

$X \subset Y$という概念は$X, Y$がともに集合であるときに定義されるものである.いま1は集合でないため,$1 \subset \{1\}$は未定義である.
ただし,$1 \not\subset \{1\}$は成り立つ,ということではないことにも注意しなければならない.$X \not\subset Y$もまた$X, Y$がともに集合であるときに定義されるからである.

\medskip

\begin{itembox}[l]{\textgt{定義 2.5} (集合の相等)}
集合$X, Y$において,$X \subset Y$かつ$Y \subset X$であるとき,またそのときのみ$X = Y$である.
\end{itembox}

\medskip

\begin{itembox}[l]{\textgt{定義 2.6} (内包関係の推移性)}
	集合$A,B,C$において
	\begin{equation}
		A \subset B, B \subset C \Rightarrow A \subset C
	\end{equation}
	が成り立つ.
\end{itembox}

\textgt{解説 2.7} 空集合$\varnothing$と任意の集合$A$の間の内包関係はどのように考えるか

空集合$\varnothing$は元を全く含まない集合であるから,どんな集合$A$に対しても$\varnothing$は$A$に含まれるとみなすのが自然である.すなわち\[\varnothing \subset A\]

また,これを示すには,\[x \in \varnothing \Rightarrow x \in A\]が成り立つことを言えばよい.しかしどんな$x$に対しても$x \not\in \varnothing$であるから$x \in \varnothing$は偽となる.よって$\varnothing \subset A$は成り立つ.

\section{11ページ問4の解答}

\begin{enumerate}
	\setcounter{enumi}{3}
	\item $a+b\sqrt{2}\ (a,b \in \mathbb{Q})$の形に表される実数全体の集合を$A$とするとき,次のことを確かめよ.
	\begin{itemize}
		\item[(i)] $x \in A, y \in A \Rightarrow x + y \in A, x - y \in A, xy \in A$\mbox{}\\
		$x = x_{1}+x_{2}\sqrt{2},\ y = y_{1}+y_{2}\sqrt{2}\ (x_{1}, x_{2}, y_{1}, y_{2} \in \mathbb{Q})$ とする.
		\begin{align}
			x + y &= x_{1}+x_{2}\sqrt{2}+ y_{1}+y_{2}\sqrt{2}\\
				  &= (x_{1}+y_{1})+(x_{2}+y_{2})\sqrt{2}
		\end{align}
		$x_{1}+y_{1},\ x_{2}+y_{2}\in \mathbb{Q}$だから,$x+y \in A$\QED
		\begin{align}
			x - y &= x_{1}+x_{2}\sqrt{2}- y_{1}+y_{2}\sqrt{2}\\
				  &= (x_{1}-y_{1})+(x_{2}-y_{2})\sqrt{2}
		\end{align}
		$x_{1}+y_{1},\ x_{2}+y_{2}\in \mathbb{Q}$だから,$x-y \in A$\QED
		\begin{align}
			x \times y &= (x_{1}+x_{2}\sqrt{2}) \times (y_{1}+y_{2}\sqrt{2})\\
					   &= x_{1}y_{1} + x_{1}y_{2}\sqrt{2} + x_{2}y_{1}\sqrt{2} + 2x_{2}y_{2}\\
					   &= (x_{1}y_{1} + 2x_{2}y_{2}) + (x_{1}y_{2} + x_{2}y_{1})\sqrt{2}
		\end{align}
		$x_{1}y_{1} + 2x_{2}y_{2},\ x_{1}y_{2} + x_{2}y_{1} \in \mathbb{Q}$だから,$x \times y \in A$\QED
		\item[(ii)] $x \in A, x \neq 0 \Rightarrow x^{-1} \in A$ \mbox{}\\
		$x = a+b\sqrt{2}\ (a,b \in \mathbb{Q})$ とする.このとき,$x \neq 0$より,$a + b\sqrt{2} \neq 0$
		\begin{align}
			x^{-1} &= \frac{1}{a+b\sqrt{2}}\\
				   &= \frac{a-b\sqrt{2}}{(a+b\sqrt{2})(a-b\sqrt{2})}\\
				   &= \frac{a-b\sqrt{2}}{a^2-2b^2}\\
				   &= \frac{a}{a^2-2b^2} + \frac{-b}{a^2-2b^2}\sqrt{2}
		\end{align}
		$\frac{a}{a^2-2b^2}, \frac{-b}{a^2-2b^2} \in \mathbb{Q}$だから,$x \in A, x \neq 0 \Rightarrow x^{-1} \in A$\QED
		\medskip
	\end{itemize}
	\item[] $a+b\sqrt{2}\ (a,b \in \mathbb{Z})$の形に表される実数全体の集合$A^{\prime}$については,上のことは成り立つか.\mbox{}\\
	(i)は同様の議論で成り立つ.
	(ii)は任意の$a,b \in \mathbb{Z}$について$\frac{a}{a^2-2b^2}, \frac{-b}{a^2-2b^2} \in \mathbb{Z}$が成り立たないため,成り立たない.
\end{enumerate}


%%参考文献%%
%\newpage
\begin{thebibliography}{99}
\bibitem{matsuzaka}松坂和夫,集合位相入門,岩波書店,2018.
\end{thebibliography}

\end{document}
